%%%%%%%%%%%%%%%%%%%%%%%%%%%%%%%%%%%%%%%%%
% University Assignment Title Page 
% LaTeX Template
% Version 1.0 (27/12/12)
%
% This template has been downloaded from:
% http://www.LaTeXTemplates.com
%
% Original author:
% WikiBooks (http://en.wikibooks.org/wiki/LaTeX/Title_Creation)
%
% License:
% CC BY-NC-SA 3.0 (http://creativecommons.org/licenses/by-nc-sa/3.0/)
% 
% Instructions for using this template:
% This title page is capable of being compiled as is. This is not useful for 
% including it in another document. To do this, you have two options: 
%
% 1) Copy/paste everything between \begin{document} and \end{document} 
% starting at \begin{titlepage} and paste this into another LaTeX file where you 
% want your title page.
% OR
% 2) Remove everything outside the \begin{titlepage} and \end{titlepage} and 
% move this file to the same directory as the LaTeX file you wish to add it to. 
% Then add \input{./title_page_1.tex} to your LaTeX file where you want your
% title page.
%t
%%%%%%%%%%%%%%%%%%%%%%%%%%%%%%%%%%%%%%%%%
\title{Template báo cáo KHTN}
%----------------------------------------------------------------------------------------
%	PACKAGES AND OTHER DOCUMENT CONFIGURATIONS
%----------------------------------------------------------------------------------------

\documentclass[12pt]{article}
\usepackage[T5]{fontenc}
\usepackage[utf8]{inputenc}
\usepackage[vietnamese,english]{babel}
\usepackage{amsmath}
\usepackage{graphicx}
\usepackage[colorinlistoftodos]{todonotes}
\usepackage{listings}
\usepackage{hyperref}
\usepackage{multicol}
\hypersetup{
    colorlinks=true,
    linkcolor=blue,
    filecolor=magenta,      
    urlcolor=cyan,
}
\setlength{\parindent}{1em}
\setlength{\parskip}{1em}
\renewcommand{\baselinestretch}{1.5}


\begin{document}

\begin{titlepage}

\newcommand{\HRule}{\rule{\linewidth}{0.5mm}} % Defines a new command for the horizontal lines, change thickness here

\center % Center everything on the page
 
%----------------------------------------------------------------------------------------
%	HEADING SECTIONS
%----------------------------------------------------------------------------------------

\textsc{\LARGE ĐẠI HỌC CÔNG NGHỆ - ĐHQGHN}\\[1.5cm]
\textsc{\Large Môn học: Chuyên Nghiệp Trong Công Nghệ}\\[0.5cm] % Major heading such as course name
\textsc{\large Nhóm 5 }\\[0.5cm] % Minor heading such as course title

%----------------------------------------------------------------------------------------
%	TITLE SECTION
%----------------------------------------------------------------------------------------

\HRule \\[0.4cm]
{ \huge \bfseries THƯƠNG MẠI ĐIỆN TỬ}\\[0.4cm] % Title of your document
\HRule \\[1.5cm]
 
%----------------------------------------------------------------------------------------
%	AUTHOR SECTION
%----------------------------------------------------------------------------------------

\begin{minipage}{1.2\textwidth}
\begin{multicols}{2}
\begin{itemize}
    \item Nguyễn Văn Huy
    \item Ngô Văn Hào
    \item Phạm Văn Hùng
    \item Nguyễn Trung Kiên
    \item Nguyễn Duy Kiên
    \item Vũ Văn Long
    \item Nguyễn Văn Mạnh
    \item Phan Văn Minh
    \item Trần Quang Minh
    \item Đặng Văn Mạnh
\end{itemize}
\end{multicols}
\end{minipage}\\[2cm]

% If you don't want a supervisor, uncomment the two lines below and remove the section above
%\Large \emph{Author:}\\
%John \textsc{Smith}\\[3cm] % Your name

%----------------------------------------------------------------------------------------
%	DATE SECTION
%----------------------------------------------------------------------------------------

% I don't want day because it is English
% {\large \today}\\[2cm] % Date, change the \today to a set date if you want to be precise

%----------------------------------------------------------------------------------------
%	LOGO SECTION
%----------------------------------------------------------------------------------------

\includegraphics[scale=0.36]{logo/UET.png}\\[2cm]\\ % Include a department/university logo - this will require the graphicx package
 
%----------------------------------------------------------------------------------------

\vfill % Fill the rest of the page with whitespace

\end{titlepage}


\section{Giới thiệu chủ đề}
Thương mại điện tử (e-commerce) là việc tiến hành một phần hay toàn bộ công việc kinh doanh bằng các phương thức điện tử. Một cách dễ hiểu, thương mại điện tử chính là việc mua bán các sản phẩm hay các dịch vụ qua Internet hoặc các phương tiện điện tử khác. Các giao dịch này bao gồm tất cả các hoạt động như: giao dịch, mua bán, thanh toán, đặt hàng, quảng cáo và giao hàng ....


Bản chất để Web và Internet phát triển trong tương lai chính là thương mại điện tử. Các “trung tâm online” sẽ ngày càng xuất hiện nhiều hơn trên Internet dưới dạng sử dụng web hay ứng dụng. Nó giúp các nhà cung cấp sản phẩm có thể tiếp cận một cách trực tiếp và nhanh chóng với người tiêu dùng. Thật vậy, ngày nay với tốc độ phát triển chóng mặt của Internet, hầu hết các công ty (nhiều nhất là các công ty về thương mại dịch vụ) đã tạo cho mình các nền tảng trên Internet nhằm lợi dụng sự phát triển của Internet để quảng bá cũng như trao đổi, mua bán sản phẩm đem lại lợi nhuận cao cho công ty. Đó là một trong những lợi ích không nhỏ của thương mại điện tử

\section{Lịch sử hình thành và phát triển của thương mại điện tử }
\subsection{Sự hình thành thương mại điện tử}
Về nguồn gốc, thương mại điện tử được xem như là điều kiện thuận lợi của các giao dịch thương mại điện tử, sử dụng công nghệ như EDI và EFT. Cả hai công nghệ này đều được giới thiệu thập niên 70, cho phép các doanh nghiệp gửi các hợp đồng điện tử như đơn đặt hàng hay hóa đơn điện tử. Sự phát triển và chấp nhận của thẻ tín dụng, máy rút tiền tự động (ATM) và ngân hàng điện thoại vào thập niên 80 cũng đã hình thành nên thương mại điện tử. Một dạng thương mại điện tử khác là hệ thống đặt vé máy bay bởi Sabre ở Mỹ và Travicom ở Anh.

Vào thập niên 90, thương mại điện tử bao gồm các hệ thống hoạch định tài nguyên doanh nghiệp (ERP), khai thác dữ liệu và kho dữ liệu.

Năm 1990, Tim Berners-Lee phát minh ra WorldWideWeb trình duyệt web và chuyển mạng thông tin liên lạc giáo dục thành mạng toàn cầu được gọi là Internet (www). Các công ty thương mại trên Internet bị cấm bởi NSF cho đến năm 1995. Mặc dù Internet trở nên phổ biến khắp thế giới vào khoảng năm 1994 với sự đề nghị của trình duyệt web Mosaic, nhưng phải mất tới 5 năm để giới thiệu các giao thức bảo mật (mã hóa SSL trên trình duyệt Netscape vào cuối năm 1994) và DSL cho phép kết nối Internet liên tục. Vào cuối năm 2000, nhiều công ty kinh doanh ở Mỹ và Châu  u đã thiết lập các dịch vụ thông qua World Wide Web. Từ đó con người bắt đầu có mối liên hệ với từ "ecommerce" với quyền trao đổi các loại hàng hóa khác nhau thông qua Internet dùng các giao thức bảo mật và dịch vụ thanh toán điện tử.


\subsection{Chèn hình}

Trích dẫn hình \ref{fig:node} trong đoạn chữ. 

% % Commands to include a figure:
\begin{figure}[h]
\centering
\includegraphics[width=0.5\textwidth]{image/node.PNG}
\caption{\label{fig:node} Nút}
\end{figure}


% chuong 3
\section{Chèn đoạn code}


ví dụ code \ref{lst:vdcode} là code python 


\begin{lstlisting}[caption={Đoạn code}, label={lst:vdcode}, language=python]
s = "I am Pusheen the cat"
print(s)
\end{lstlisting}

ví dụ trích dẫn \cite{robinson2013graph}


\bibliographystyle{IEEEtran}
\bibliography{bib}

%%%%%%%%%%%%%%%%%%%%%%%%%%%%%%%%%%%%%%%%%%%%%%%%%%%%
% Comments can be added to the margins of the document using the \todo{Here's a comment in the margin!} todo command, as shown in the example on the right. You can also add inline comments too:

% \todo[inline, color=green!40]{This is an inline comment.}



% \subsection{Tables and Figures}

% Use the table and tabular commands for basic tables --- see Table~\ref{tab:widgets}, for example. You can upload a figure (JPEG, PNG or PDF) using the files menu. To include it in your document, use the includegraphics command as in the code for Figure~\ref{fig:frog} below.

% % % Commands to include a figure:
% % \begin{figure}
% % \centering
% % \includegraphics[width=0.5\textwidth]{frog.jpg}
% % \caption{\label{fig:frog}This is a figure caption.}
% % \end{figure}

% % \begin{table}
% % \centering
% % \begin{tabular}{l|r}
% % Item & Quantity \\\hline
% % Widgets & 42 \\
% % Gadgets & 13
% % \end{tabular}
% % \caption{\label{tab:widgets}An example table.}
% % \end{table}

% \subsection{Mathematics}

% \LaTeX{} is great at typesetting mathematics. Let $X_1, X_2, \ldots, X_n$ be a sequence of independent and identically distributed random variables with $\text{E}[X_i] = \mu$ and $\text{Var}[X_i] = \sigma^2 < \infty$, and let
% $$S_n = \frac{X_1 + X_2 + \cdots + X_n}{n}
%       = \frac{1}{n}\sum_{i}^{n} X_i$$
% denote their mean. Then as $n$ approaches infinity, the random variables $\sqrt{n}(S_n - \mu)$ converge in distribution to a normal $\mathcal{N}(0, \sigma^2)$.

% \subsection{Lists}

% You can make lists with automatic numbering \dots

% \begin{enumerate}
% \item Like this,
% \item and like this.
% \end{enumerate}
% \dots or bullet points \dots
% \begin{itemize}
% \item Like this,
% \item and like this.
% \end{itemize}

% We hope you find write\LaTeX\ useful, and please let us know if you have any feedback using the help menu above.

\end{document}